\chapter {Dice game}
\section{Project description}

The purpose of the second project was to analyse die rolls for the purpose of playing a popular dice game called ``Yahtzee''. The game consists of rolling 5 dies a maximum of 3 times per turn. Desirable rolls are kept aside and the remainder is cast again. The objective is to score as many points as possible by placing the final rolls in one of 15 categories. 

In particular, it was attempted to be able to read off the die rolls with a single camera. A number of existing methods were found which utilised multiple cameras to increase the accuracy of the readings \cite{hsu2012color}, for example in widely varying lighting conditions. As the lighting conditions could be controlled in this project, only a single camera was chosen. 

Additionally, two constraints were added. First, the camera is allowed to be positioned at skew angles, and not directly above the scene. Second, the size of the dies is not given. This is in effect similar to allowing the camera to be placed at different heights.

\section{Method}
\subsection{Overview}
The general methodology that was adopted for extracting the die rolls from the image taken by the camera was as follows:
\begin{enumerate}
\item Segment the image into areas that might represent dots and background pixels.
\item Deduce which of those areas actually represent dots.
\item Deduce which dots belong to the same die face, thus determining the roll of the die.
\end{enumerate}

Each of these steps will be discussed and supported in detail.

The project uses a set of standard dice. In particular, white dies with black dots. For the remainder of this report, whenever a feature of a die is mentioned it is considered a feature most dice in existence posess.

\subsection{Image segmentation step}
The objective of this step is to obtain a binary image which solely contains information related to dots so that inforation about them can be extracted in the subsequent step. The aforementioned high-contrast colouring of standard dice is a very useful feature for segmentation as it is relatively easy to exploit.

\subsubsection{Problems which complicate segmentation}
Unfortunately, dice also exhibit some features which significantly complicate the segmentation process:

\begin{description}
\item[Rounded off corners] In order to encourage dice to roll before finally landing on a surface the corners and edges of dice are rounded off. This significantly complicates the detection process of die faces, as colours change gradually rather than instantly from one side to the next. This makes it much harder to find out which face a detected dot belongs to. Figure \ref{fig:diesEdges} shows a die together with an object with sharp corners. It is easier to extract data about different surfaces on the object with sharp corners because the rapid changes in colour can be exploited. In the case of dice this is much more difficult.

\begin{figure}
	\centering
	\includegraphics[width=50mm]{images/dies/edges.jpg}
	\caption{A single die and a shape with sharp edges.}
	\label{fig:diesEdges}
\end{figure}

\item[High specular factor] Dice are commonly polished. This surface therefore has a high specular factor which in turn results in many specular reflections that are picked up by the camera. 
\item[Dots are not flat] To allow blind people to read off their die rolls, the dots on dies are slightly hollowed out. Due to their shape an additional specular reflection is created on each dot. On this image this is represented as a much lighter coloured area inside an area which should preferrably be black completely. Figure \ref{fig:dotSpecular} shows the effect such specular reflections can have. The segmented image clearly shows reduced performance because of the brighter colours inside the die dots.

\begin{figure}
	\centering
	\begin{subfigure}[b]{3cm}
		\includegraphics[width=30mm]{images/dies/specular.png}
		\caption{Original}
		\label{fig:dotSpecularOriginal}
	\end{subfigure}
	%
	\hspace{1cm}
	%
	\begin{subfigure}[b]{3cm}
		\includegraphics[width=30mm]{images/dies/specular_thresholded.png}
		\caption{Segmented}
	\end{subfigure}

	\caption{A single die showing specular reflections within its dots, and the result of its segmentation.}
	\label{fig:dotSpecular}
\end{figure}

\item[Shadows] In figure \ref{fig:dotSpecularOriginal} can also be seen that shadows may contain shadows. These can be quite dark depending on the lighting conditions. They may thus be picked up by the segmentation algorithm as possible dots. 
\end{description}

\subsubsection{Developed method}
The method used in the final product is the following sequence of filters:

\begin{description}
\item[Gaussian blur (9x9 kernel)] Used to reduce the noise level going into the filtering process. The used camera showed significant amounts of noise in low-light conditions. Aplying a blur on beforehand significantly increased the quality of the dot recognition.

\begin{figure}
	\centering
	\begin{subfigure}[b]{45mm}
		\includegraphics[width=45mm]{images/dies/original.png}
		\caption{Original}
	\end{subfigure}
	%
	\hspace{0.5cm}
	%
	\begin{subfigure}[b]{45mm}
		\includegraphics[width=45mm]{images/dies/dilated.png}
		\caption{Dilated}
	\end{subfigure}
	%
	\hspace{0.5cm}
	%
	\begin{subfigure}[b]{45mm}
		\includegraphics[width=45mm]{images/dies/difference.png}
		\caption{Difference}
		\label{fig:imgdiff}
	\end{subfigure}

	\caption{The bottom-hat filtering process.}
	\label{fig:bottomhat}
\end{figure}

\item[Bottom-Hat (11x11 square kernel)] Method from lecture 6, slide 42. This is the most important part of the filter sequence. Its main objective is to increase the contrast between the dice and the die faces while simultaniously reducing much of the background noise. The results of the individual processing steps are shown in figure \ref{fig:bottomhat}. The initial dilation causes the dots to disappear. Calculating the difference after an erosion step gives a nice segmentation of the dots as shown in figure \ref{fig:imgdiff}. The dots on the dice are clearly visible.

Also note that most of the sharp shadows have been taken out, as has the background which in itself was not a noise-free surface. However, some of the thinner shadows and darker surfaces have also come through.

The Bottom-Hat filter was chosen because of its ability to filter out high-contrast features on an image while simultaneously reducing background noise. A number of filters have been tried on this problem, though this method came out superior.

\item[Threshold (Otsu's method)] Finally, the image is converted into a binary image using a thresholding filter. The threshold level is determined using Otsu's method, as the image clearly contains two distinct sets of brighter and darker pixels. Otsu's method is also stronger in adapting to different lighting conditions as the intensity of the pixels from the original camera frame can vary. A single constant threshold was thus not satisfactory for this purpose.

\end{description}

\subsubsection{Attempted alternate methods}

\begin{figure}
	\centering
	\includegraphics[width=40mm]{images/dies/canny_2.png}
	\caption{A canny edge detector applied on a captured image frame.}
	\label{fig:cannyEdge}
\end{figure}

\begin{description}
	\item[Canny edge detector] A more direct approach that was considered was to use a canny edge detector to locate the edge between a die dot and its background. Which is the followed up by a detector of ellipses to classify the individual dots. This method had some promising results, as shown in figure \ref{fig:cannyEdge}. However, the method described above was chosen in favour of it because it did not discard information about the location of darker patches while simultaniously being able to extract a similar boundary in its later stages (see figure \ref{fig:contour}. As this gave an overall higher information level, the edge detection method was not used.
	\item[Dynamic threshold] It was attempted to use a dynamic threshold in favour of a global one. In particular a gaussian threshold. This unfortunately only served as a means to amplify the background noise while simultaniously giving comparable results for the binarisation of dots to when Otsu's method was used.
\end{description}

\subsection{Dot detection step}
\subsubsection{Complications}

\begin{figure}
	\centering
	\includegraphics[width=30mm]{images/dies/ellipses.jpg}
	\caption{Image of a die taken at a 45 degree angle.}
	\label{fig:ellipse}
\end{figure}

The main complication in this step that was found during the development of the program was the issue of detecting which face a dot belonged to. Figure \ref{fig:ellipse} shows an image of a die taken at a 45 degree angle. As the dots on the die have approximately the same radius, taking a picture at this angle causes all dots to look similar. It is at this point not possible to deduce whether a dot is front-facing or top-facing with a single camera. Moreover, the noise level in the image and the resolution of the used camera significantly limited the reliability of the detected dot areas. This in turn meant it was not possible to detect the facing of a dot in the first place.

The common solution to this problem in literature appears to be to use multiple cameras (for example: \cite{hsu2012color}), or to place the camera directly above the dice so the any other sides are not visible \cite{lapanja2000computer} \cite{correia1995automated}. As the former was not an option, the only possible solution was to opt for the latter. As such the requirement for multiple angles had to be discarded as it proved simply too complicated to implement within the time frame of the project.

\subsubsection{Developed method}

When the image has been segmented into possible dots and background noise, the next step is to find all said areas and decidec whether it represents a dot or not. 

In order to extract all applicable regions initially the region labelling algortithm described in lecture 9, slide 14 was used. Despite its efficiency it proved difficult for the remaining implementation to handle the equivalent classes. Instead a somewhat less efficient flood fill algorithm was used. 

Next, all areas below 15 pixels in size are discarded. As these are commonly bits of noise rather than actual die dots, they can be discarded right away. These areas also had a tendency to be classified as valid dots, thus a minimum size was required for good results.

For the remaining areas the center of mass is calculated along with their contour. The result is shown in figure \ref{fig:contour}. Next, the closest and furthest contour pixel from the center of mass is deduced. A ratio is calculated from them, and if below a certain threshold, the area is accepted as a dot. Note that the pixels that did not belong to the area contour still provided useful information in this step, even though the canny edge detector would have principally skipped a number of steps up to this point.

\begin{figure}
	\centering
	\includegraphics[width=60mm]{images/dies/contour.png}
	\caption{Detected contours extracted projected on to the original frame.}
	\label{fig:contour}
\end{figure}

The idea of this method is shown in figure \ref{fig:minmax}. As the camera is now assumed to be directly over the dice, the dots should appear to be almost perfectly circular. This in turn means that the pixel closest to the area's center of mass should approximately be the same as the one furthest away. A ratio was adopted to allow this detection method to work on different resolutions and die sizes. It resulted in decent detection results in practice. However, it still experienced problems with the specular reflections of die dots. The result of a classification is shown in figure \ref{fig:acceptedRejected}.

\begin{figure}
	\centering
	\includegraphics[width=60mm]{images/dies/minmax.png}
	\caption{An illustration of the dot classification process.}
	\label{fig:minmax}
\end{figure}

\begin{figure}
	\centering
	\includegraphics[width=60mm]{images/dies/rejectedAreas2.png}
	\caption{Accepted and rejected areas on top of the original input frame.}
	\label{fig:acceptedRejected}
\end{figure}

\subsection{Die roll classification step}
The previously centers of mass of each area can now be reused as a means of grouping dots together which belong to the same die face. The method employed for this was to link up all dots within a certain threshold. A more advanced method using pattern matching was described in \cite{correia1995automated}, but was not used due to its complexity and time constraints. 

The major downside of this method is that it is unable to make use of the fact that dies only contain 6 dots at most and it is not scalable as it uses a pixel threshold. However, it proved adequate when the dice were slightly spaced out. Figure \ref{fig:filteringComplete} shows the final result of the described filtering process.

\begin{figure}
	\centering
	\includegraphics[width=60mm]{images/dies/rollRecognition.png}
	\caption{The result of the filtering process. Blue lines indicate which dots belong to the same die face.}
	\label{fig:filteringComplete}
\end{figure}

It would have been useful here to include information about the location of the white die backgrounds into the roll detection process. Unfortunately extracting these die backgrounds can prove difficult as its success depends on the surface the dies are located at. For example: a white surface makes it very difficult to extract the contours of a die which itself is white. It may be possible to employ a watershed segmentation to distinguish the dice from each other. 

\section{Conclusion}
The developed system has turned out sufficient albeit not very robust. It was able to detect die rolls correctly on a number of background surfaces and varying lighting conditions. The implementation had most problems with the specular reflections in die dots. Due to the relatively strict dot classification rules they proved difficult to detect properly in some situations. It was, however, successful in discarding the shadows whenever they came through the bottom-hat filtering step. 